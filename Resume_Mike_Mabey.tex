%%%%%%%%%%%%%%%%%%%%%%%%%%%%%%%%%%%%%%%%%
% Medium Length Professional CV
% LaTeX Template
% Version 2.0 (8/5/13)
%
% This template has been downloaded from:
% http://www.LaTeXTemplates.com
%
% Original author:
% Trey Hunner (http://www.treyhunner.com/)
%
% Important note:
% This template requires the resume.cls file to be in the same directory as the
% .tex file. The resume.cls file provides the resume style used for structuring the
% document.
%
%%%%%%%%%%%%%%%%%%%%%%%%%%%%%%%%%%%%%%%%%

%----------------------------------------------------------------------------------------
%	PACKAGES AND OTHER DOCUMENT CONFIGURATIONS
%----------------------------------------------------------------------------------------

\documentclass{resume} % Use the custom resume.cls style

% Doc margins are defined in resume.cls so the footer displays properly. Can't figure out why this is even necessary, but if these values change, so should the ones in resume.cls.
%\usepackage[left=0.75in,top=1in,right=0.75in,bottom=1in]{geometry} % Document margins

% Set font to sans serif
\renewcommand*{\familydefault}{\sfdefault}

\usepackage{multicol} %columns for lists
\usepackage{paralist} %lists without indentation
\usepackage{hyperref} %links - http and mailto
\hypersetup{hidelinks} %hide boxes around links
%check out other hyperref options

\usepackage{microtype}

\setlength\multicolsep{0pt} %remove space before and after multicol

\def\sectionskip{\smallskip} % The space after the heading section
\def\nameskip{\medskip} % The space after your name at the top
\def\addressskip{} % The space between the two address (or phone/email) lines

\name{Michael K. Mabey} % Your name
\address{1037 E Libra Dr \\ Tempe, AZ 85283} % Your address
\address{\href{http://mikemabey.com}{mikemabey.com}}
\address{(480)~788--3411 \\ \href{mailto:mmabey@asu.edu}{mmabey@asu.edu}}

\icon{0.5in}{LinkedInQR2}

\begin{document}

%----------------------------------------------------------------------------------------
%	EDUCATION SECTION
%----------------------------------------------------------------------------------------

\begin{rSection}{Education}

\textbf{PhD Computer Science (Information Assurance)}\hfill \emph{Dec 2016}\\
Arizona State University $\mid$ GPA 3.74\hfill {Tempe, AZ}

\textbf{M.S. Computer Science (Information Assurance)}\hfill \emph{Aug 2011}\\
Arizona State University $\mid$ GPA 3.58\hfill {Tempe, AZ}\\
\textit{Thesis:} \href{http://repository.asu.edu/attachments/56996/content/Mabey_asu_0010N_10959.pdf}{Collaborative Digital Forensics: Architecture, Mechanisms, and Case Study}

\textbf{B.S. Computer Science (Information Systems)}\hfill \emph{May 2009}\\
Utah State University $\mid$ GPA 3.25\hfill {Logan, UT}

\end{rSection}


%----------------------------------------------------------------------------------------
%	WORK EXPERIENCE SECTION
%----------------------------------------------------------------------------------------

\begin{rSection}{Experience}

%\begin{rSubsection}{1: Company}{2: Dates}{3: Title}{4: Location}
%\end{rSubsection}

\begin{rSubsection}{Research Assistant~}{Nov 2009 -- Present}{\href{http://sefcom.asu.edu/}{Security Engineering for Future Computing (SEFCOM) Lab, ASU}}{Tempe, AZ\\{\textnormal{\textit{Advisor}: Prof. Gail-Joon Ahn\hfill \textit{Sponsors}: Department of Energy and National Science Foundation}}}
	
	\item Recipient of the Department of Defense Information Assurance Scholarship Program (IASP) for the 2012--2013, 2013--2014, and 2014--2015 school years.
	
	\item Researched forensic techniques for \textbf{Chrome~OS}/\textbf{Chromium~OS} by creating tools in \textbf{Python} and using \textbf{libvirt} to interface with \textbf{KVM/QEMU}. (Project in progress.)
	
	\item Developed a web email acquisition approach that reestablishes persistent cookie sessions stored by a browser, and automated the process using \textbf{Python} and \textbf{Selenium}~\cite{Paglierani2013}.
	%Wrote an accompanying \textbf{Python} script that used \textbf{Selenium} to then interact with the online email account.
	
	\item Designed and implemented the core components of a modular, highly scalable, collaboration-centric digital forensic framework built on the \textbf{OpenStack} cloud architecture. Functions of the components included distributed job scheduling, storage management, and concise evidence representation and transmission~\cite{Mabey2011a,Mabey2011}.
	
	\item Set up and maintained an \textbf{OpenVPN} installation for the SEFCOM lab.
		
	%\item Researched methods for performing forensic acquisition on \textbf{Android} devices.
	
	%\item Acted as a mentor for an undergraduate student that otherwise would not have pursued a master's degree and collaborated with him on the research for his thesis.
	
\end{rSubsection}

%------------------------------------------------

\begin{rSubsection}{Teaching Assistant}{Aug 2010 -- Present}{Arizona State University}{Tempe, AZ}
	
	\item[] \vspace{-1.6em}
	\end{list}
	
	\emph{Instructor positions:}
	\begin{list}{$\cdot$}{\leftmargin=0em} % \cdot used for bullets, no indentation
		\itemsep -0.5em \vspace{-0.5em} % Compress items in list together for aesthetics
	\item CSE 465 Information Assurance: Fall 2014.
	\item FSE 100 Introduction to Engineering: Fall 2011, Spring 2012, Fall 2012, Spring 2013, and Fall 2013.
	\end{list}
	
	\emph{Assistant positions:}
	\begin{list}{$\cdot$}{\leftmargin=0em} % \cdot used for bullets, no indentation
		\itemsep -0.5em \vspace{-0.5em} % Compress items in list together for aesthetics
	\item CSE 469 Computer and Network Forensics with Dr. Gail-Joon Ahn: Spring 2015.
	\item FSE 100 Introduction to Engineering with Dr. Ryan Meuth: Spring 2014.
	\item CSE 423/424 Capstone I and CSE 485/486 Capstone II with Dr. Debra Calliss: Spring 2014.
	\item CSE 467 Data \& Information Security with Dr. Gail-Joon Ahn: Spring 2011.
	\item CSE 465 Information Assurance with Dr. Gail-Joon Ahn: Fall 2010.
	
	%\item Instructor for CSE 465 Information Assurance: Fall 2014
	
	%\item Instructor for FSE 100 Introduction to Engineering: Fall 2011, Spring \& Fall 2012, Spring \& Fall 2013.

	%\item Spring 2014: Assistant to Dr. Ryan Meuth for FSE 100 Introduction to Engineering.
	
	%\item Spring 2014: Assistant to Dr. Debra Calliss for CSE 423/424 Capstone I and CSE 485/486 Capstone II.
	
	%\item Spring 2011: Assistant to Dr. Gail-Joon Ahn for CSE 467 Data \& Information Security.
	
	%\item Fall 2010: Assistant to Dr. Gail-Joon Ahn for CSE 465 Information Assurance.
	
\end{rSubsection}

%------------------------------------------------

\begin{rSubsection}{Student Trainee}{Jun 2013 -- Aug 2013, Jun 2014 -- Aug 2014}{US Army}{Fort Meade, MD}


	\item Summer internships in connection with DoD IASP scholarship.

\end{rSubsection}

%------------------------------------------------
\clearpage

\begin{rSubsection}{Graduate Student Summer Intern}{May 2011 -- Jul 2011}{Sandia National Laboratories}{Albuquerque, NM}

	\item Helped design a dynamic malware analysis framework built on \textbf{OpenStack}, allowing incident responders to define customizable analysis environments and use arbitrary analysis tools in triage or manual analysis mode.
	
	\item Wrote \textbf{Python} scripts to automate the setup process for using a SheevaPlug computer as a wireless intrusion detection agent running \textbf{Kismet}.

\end{rSubsection}

%------------------------------------------------

%\begin{rSubsection}{Graduate Student Recruitment Specialist/Webmaster}{Jul 2006 -- Aug 2009}{Electrical \& Computer Engineering Department USU}{Logan, UT}
%
%	\item Primary responsibilities included maintaining and augmenting the department website using \textbf{PHP}, \textbf{MySQL}, and other basic web technologies like \textbf{CSS}, \textbf{JavaScript}, and an \textbf{SMTP} server.
%	
%	\item Replaced a \textbf{MS Access} database by porting the old data to a \textbf{MySQL} server and creating a set of \textbf{Python} programs with the \textbf{Dabo} framework that interfaced with the database.
%	
%	%\item Wrote \textbf{Python} scripts to convert tab-delimited and Excel formatted data to \textbf{SQL} entries.
%	
%	%\item Created a testing environment using an \textbf{Apache} web server, \textbf{PHP}, \textbf{MySQL}, and \textbf{SVN}.
%	
%	%\item Gathered statistics on web visitors, inquiries from students, and applicants' credentials for the purpose of improving graduate student recruitment processes for the department.
%	
%	%\item Responded to inquiries from potential domestic and international graduate students.
%
%\end{rSubsection}

\end{rSection}

%----------------------------------------------------------------------------------------
%	PUBLICATIONS SECTION
%----------------------------------------------------------------------------------------

\bibliographystyle{resumepubs}
\mybibliography{mypapers}

%----------------------------------------------------------------------------------------
%	TECHNICAL STRENGTHS SECTION
%----------------------------------------------------------------------------------------

\begin{rSection}{Technical Strengths \& Qualifications}

\begin{tabular}{ @{} >{\bfseries}l @{\hspace{6ex}} l }
Research Interests & Digital Forensics, Internet of Things, Cloud Computing \\
Programming Languages & Python, C/C++, HTML, CSS, Answer Set Programming \\
Forensic Tools & FTK, Sleuth Kit \& Autopsy, dd, HxD, etc.\\
Protocols \& APIs & JSON, XML, REST \\
Network Administration/Security & OpenVPN, Wireshark \\
Operating Systems & Windows, Linux, Chrome OS \\
Databases & MySQL, SQLite 
\end{tabular}

\end{rSection}

%----------------------------------------------------------------------------------------
%	SCHOOL PROJECTS SECTION
%----------------------------------------------------------------------------------------

%\begin{rSection}{Relevant School Projects}
%\begin{list}{$\cdot$}{\leftmargin=0em} % \cdot used for bullets, no indentation
%   \itemsep -0.5em
%
%	\item Implemented \textbf{Python} scripts that interpreted MBRs and boot sectors for FAT file systems.
%	
%	\item Wrote a program in \textbf{Python} to scan \textbf{C} and \textbf{C++} source files for commonly used but insecure function calls, which then suggested to the user more secure yet equivalent library functions.
%	
%	%\item Implemented a ``robot'' for the Google Wave platform that translated hex and binary to ASCII as well as interpreted MBRs and VBRs. See http://code.google.com/p/forensie/.
%	
%	\item Created a web-based interactive learning module designed to teach basic principles of password strength and symmetrical encryption using \textbf{Python}, \textbf{JavaScript}, and \textbf{Ajax}. \href{http://csilm.usu.edu/~securityninja/}{http://csilm.usu.edu/\textasciitilde securityninja/}.
%	
%	%\item Wrote a program in \textbf{C++} that accepted a file containing cipher text created with a symmetrical encryption algorithm and attempted to discover the original key and plain text.
%	
%	%\item Demonstrated a buffer overflow attack on a \textbf{C++} program using \textbf{gdb}.
%	
%	%\item Configured a \textbf{Linux} machine to be a \textbf{DNS} server, a \textbf{DHCP} server, and a web server that used \textbf{SSL} and authentication via \textbf{htaccess} rules.
%	
%	%\item Set up a basic logging system in \textbf{Linux} with \textbf{Snort} and \textbf{syslog}.
%	
%	%\item Created a map of local wireless networks by war-walking while running \textbf{NetStumbler}.
%	
%	%\item Analyzed Internet Explorer vulnerabilities using Security Innovation's \textbf{Holodeck} program.
%
%\end{list}
%
%\end{rSection}

%----------------------------------------------------------------------------------------
%	AWARDS SECTION
%----------------------------------------------------------------------------------------

\begin{rSection}{Awards and Activities}

\begin{list}{$\cdot$}{\leftmargin=0em} % \cdot used for bullets, no indentation
   \itemsep -0.5em
	
	\item DoD Information Assurance Scholarship Program (IASP) Recipient (3 years) \hfill 2012--2015

	\item Inducted into Eta Kappa Nu (HKN) Engineering Honors Society \hfill Nov 2010
	
	\item Team Leader --- ASU team in the UCSB International CTF \hfill 2009, 2010, 2015
	
	\item Eagle Scout, Boy Scouts of America \hfill 2002
	
	\item Tallest Graduate Student at ASU \hfill (not an actual award)

\end{list}

\end{rSection}

%----------------------------------------------------------------------------------------
%	EXAMPLE SECTION
%----------------------------------------------------------------------------------------

%\begin{rSection}{Section Name}

%Section content\ldots

%\end{rSection}

%----------------------------------------------------------------------------------------

\end{document}
