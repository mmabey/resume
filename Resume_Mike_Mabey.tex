%%%%%%%%%%%%%%%%%%%%%%%%%%%%%%%%%%%%%%%%%
% Medium Length Professional CV
% LaTeX Template
% Version 2.0 (8/5/13)
%
% This template has been downloaded from:
% http://www.LaTeXTemplates.com
%
% Original author:
% Trey Hunner (http://www.treyhunner.com/)
%
% Important note:
% This template requires the resume.cls file to be in the same directory as the
% .tex file. The resume.cls file provides the resume style used for structuring the
% document.
%
%%%%%%%%%%%%%%%%%%%%%%%%%%%%%%%%%%%%%%%%%

%----------------------------------------------------------------------------------------
%	PACKAGES AND OTHER DOCUMENT CONFIGURATIONS
%----------------------------------------------------------------------------------------

\documentclass{resume} % Use the custom resume.cls style

% Set font
% Computer Modern Sans Serif - the default sans serif font. Pretty good, but not great.
% \renewcommand*{\familydefault}{\sfdefault}
% \usepackage[T1]{fontenc}

% Bera Sans - Really like this one :)
% \usepackage[]{berasans}
% \renewcommand*\familydefault{\sfdefault}
% \usepackage[T1]{fontenc}

% Cantarell - really like this one, the letter l is easier to distinguish from its little tail
\usepackage[default, scale=0.95]{cantarell}
\usepackage[T1]{fontenc}

% Fira - too sloppy, smooshed

% Lato - I kind of like this one
% \usepackage[default, scale=0.95]{lato}
% \usepackage[T1]{fontenc}

% Latin Modern SansLatin Modern Sans - weird rendering of certain symbols, otherwise very similar to default (computer modern)

% Merriweather Sans Light - don't really like this

% Open Sans - bold is too thick, words too close together

% Paratype Sans - similar to Cantarell, but not as nice

% Quattrocento Sans - Like this one, small trade-offs from Cantarell
% \usepackage[sfdefault]{quattrocento}
% \usepackage[T1]{fontenc}

% Roboto - doesn't look professional enough

% Source Sans Pro - I like this, but don't think I like it better than Cantarell
% \usepackage[default,semibold]{sourcesanspro}
% \usepackage[T1]{fontenc}

% TEX Gyre Heros - doesn't look professional enough, text is kind of bunched

% Universalis ADF Standard - pretty good, not in love with M character

% URW A030 - Very close to arial, but in a bad way

% URW Nimbus Sans - pretty good, bold is super thick, size is kind of small, resembles Helvetica, size is a little better without the [scaled] option

% Venturis ADF Sans - not sure if it's readable enough, looks like it has serifs... weird
% \usepackage[lf]{venturis} %% lf option gives lining figures as default;
%         %% remove option to get oldstyle figures as default
% \renewcommand*\familydefault{\sfdefault}
% \usepackage[T1]{fontenc}



% Flag for switching off content specific to CV (extended) format
\setboolean{isCV}{false}

\name{Michael K.\ Mabey} % Your name
\footername*{Mabey}
\address{%
  (480)\,788--3411 \\ %
  \href{mailto:mmabey@ieee.org}{mmabey@ieee.org} \\ %
  \href{https://mikemabey.com}{mikemabey.com}%
}
\address{%
  \href{https://github.com/mmabey/}{\faGithub\ github.com/mmabey} \\ %
  \href{https://bitbucket.org/mmabey/}{\faBitbucket\ bitbucket.org/mmabey}%
}

\icon{0.5in}{figs/LinkedInQR2}



% Add .bib files as resources for biblatex
\addbibresource{pubs_mike_mabey.bib}


\begin{document}

% !TEX root = ../CV_Mike_Mabey.tex

%----------------------------------------------------------------------------------------
%	EDUCATION SECTION
%----------------------------------------------------------------------------------------

\begin{rSection}{Education}

\textbf{Ph.D.\ Computer Science --- Information Assurance}\hfill \emph{(expected) Dec 2017}\\
Arizona State University\rResumeOnly{ | GPA 3.76} \hfill {Tempe, AZ}\\
\textit{Committee:} Gail-Joon Ahn (Co-Chair), Adam Doup\'{e} (Co-Chair), Stephen S.\ Yau, Jooyung Lee, Ziming Zhao\\
\textit{Research Topics:} Web environment forensics, Email forensics, Chrome OS forensics

\textbf{M.S.\ Computer Science --- Information Assurance}\hfill \emph{Aug 2011}\\
Arizona State University\rResumeOnly{ | GPA 3.58} \hfill {Tempe, AZ}\\
\textit{Committee:} Gail-Joon Ahn (Chair), Stephen S.\ Yau, Dijiang Huang\\
\textit{Thesis:} \href{http://repository.asu.edu/attachments/56996/content/Mabey_asu_0010N_10959.pdf}{Collaborative Digital Forensics: Architecture, Mechanisms, and Case Study}\begin{CVonly}
\begin{quoting}
  In order to catch the smartest criminals in the world, digital forensic examiners need a means of collaborating and sharing information that is not prohibitively difficult to use and that complies with standard operating procedures and the rules of evidence. In this thesis I present the design and implementation of the Collaborative Forensic Framework (CUFF) which is a cloud-based forensic analysis environment that facilitates collaboration among examiners and their trusted colleagues. CUFF makes a positive impact on forensic examiners' efficiency by helping them leverage their network of subject matter experts while also tapping into the scalability and power of the cloud.
  % I also give the details of a realization of CUFF, which uses a combination of Java, the Google Web Toolkit, Django with Apache for a RESTful web service, and an Ubuntu Enterprise Cloud using Eucalyptus.
\end{quoting}\end{CVonly}

\textbf{B.S.\ Computer Science --- Information Systems}\hfill \emph{May 2009}\\
Utah State University\rResumeOnly{ | GPA 3.25} \hfill {Logan, UT}

\end{rSection}



%----------------------------------------------------------------------------------------
%	WORK EXPERIENCE SECTION
%----------------------------------------------------------------------------------------

\begin{rSection}{Experience}

  
% Define the fields for this entry
\expPosTitle{Research Assistant}
\expPosPeriod{Nov 2009 -- Present}
\expOrgName{\href{http://sefcom.asu.edu/}{Security Engineering for Future Computing (SEFCOM) Lab}, ASU}
\expOrgLoc{Tempe, AZ}
\expExtra{\textit{Lab Directors}: Gail-Joon Ahn, Adam Doup\'{e}, Ziming Zhao%
  \rResumeOnly{\\ \textit{Sponsors}: Department of Energy, National Science Foundation}% CV lists sponsors down below
  }


\begin{rExperience}

  \rCVonly{ % CV gives a list of projects

    \item Projects:

      \begin{rBulletList}
  }

        \item Created a method for identifying extensions installed on \textbf{Chrome~OS} by analyzing the encrypted
          files on the hard drive. Wrote an accompanying crawler in \textbf{Python} (and using \textbf{Ansible},
          \textbf{Celery}, \textbf{MySQL}, \textbf{sshfs}, and \textbf{OpenStack}) to download all extensions on the
          Chrome Web Store and analyze them.\rCVonly{\\
          \emph{Publications:}~\cite{MabeyDFRWS, IASymp_Mabey2015, MabeyBlogCheckout2016, MabeyBlogFixRepo2015}}

        \rCVonly{
        \item Helped design and implement a cloud-based version of the International Capture the Flag (iCTF)
          competition, allowing educators to more easily host their own CTF competitions. Used \textbf{Ansible},
          \textbf{Vagrant}, \textbf{Amazon EC2}, and \textbf{Python} for the implementation and deployment.\\
          \emph{Publications:}~\cite{Trickel2017}
        }

        \item Developed a forensic acquisition approach for web email that reestablishes persistent cookie sessions
          stored by a browser, and automated the process using \textbf{Python} and \textbf{Selenium}.\rCVonly{\\
          \emph{Publications:}~\cite{Paglierani2013}}

        \item Designed and implemented the core components of a modular, highly scalable, collaboration-centric digital
          forensic framework built on the \textbf{OpenStack} cloud architecture. Functions of the components included
          distributed job scheduling, storage management, and concise evidence representation and
          transmission.\rCVonly{\\
          \emph{Publications:}~\cite{Mabey2011, Mabey2013, IAWS_Mabey2012}}

  \rCVonly{

      \end{rBulletList}

    \item Other experience:
      \begin{rBulletList}
  }

        \item Maintained fifteen servers for the lab, including a public-facing router, an \textbf{OpenVPN} server, a
          reverse-proxy web server with \textbf{TLS} certificate management, an \textbf{OpenStack} cloud, switches
          transmitting \textbf{VLAN}-tagged traffic, and a \textbf{GitLab} server.

  \rCVonly{

        % \item Acted as a mentor for an undergraduate student that otherwise would not have pursued a master's degree and
          % collaborated with him on the research for his thesis.

      \end{rBulletList}

    \item Sponsors:
      \begin{rBulletList}[2]

        \item Department of Defense Information Assurance Scholarship Program (IASP)

        \item Department of Energy

        \item National Science Foundation

      \end{rBulletList}

  }

\end{rExperience}


  
% Define the fields for this entry
\def\PositionTitle{Teaching Assistant}
\def\PositionPeriod{Aug 2010 -- Dec 2015}
\def\OrgName{Arizona State University}
\def\OrgLocation{Tempe, AZ}


\ifthenelse{\boolean{titleOnly}}{
  \begin{rExperienceHeader}{\PositionTitle}{\PositionPeriod}{\OrgName}{\OrgLocation}
  \end{rExperienceHeader}
}{
\begin{rExperienceBullets}{\PositionTitle}{\PositionPeriod}{\OrgName}{\OrgLocation}

  %\item CSE 465 Information Assurance with Dr. Gail-Joon Ahn: Fall 2015, Fall 2010.
  \item CSE 465 Information Assurance with Dr.\ Gail-Joon Ahn: Fall 2010, Fall 2015.

  \item CSE 469 Computer and Network Forensics with Dr.\ Gail-Joon Ahn: Spring 2015.

  \item FSE 100 Introduction to Engineering with Dr.\ Ryan Meuth: Spring 2014.

  \item CSE 423/424 Capstone I and CSE 485/486 Capstone II with Dr.\ Debra Calliss: Spring 2014.

  \item CSE 467 Data \& Information Security with Dr.\ Gail-Joon Ahn: Spring 2011.

\end{rExperienceBullets}

}


  
% Define the fields for this entry
\expPosTitle{SAT/ACT Instructor}
\expPosPeriod{Sep 2015 -- Oct 2015} % Remember to use -- between dates
\expOrgName{Minerva Learning, LLC}
\expOrgLoc{\iftoggle{fullAddress}{905 N McClintock Dr, Chandler, AZ 85226}{Chandler, AZ}}
\iftoggle{superInfo}{\expExtra{%
  \textit{Supervisor:} Heidi Manoguerra, (480)~297-6430, \href{mailto:heidilkm@gmail.com}{heidilkm@gmail.com}}}{}


\begin{rExperience}

  \item Individual tutor for a high school student preparing for the PSAT.

\end{rExperience}


  
% Define the fields for this entry
\expPosTitle{Summer Intern}
\expPosPeriod{Jul 2015 -- Sep 2015} % Remember to use -- between dates
\expOrgName{Arizona Cyber Threat Response Alliance (ACTRA)}
\expOrgLoc{\iftoggle{fullAddress}{16212 N 28 Ave \#100, Phoenix, AZ 85053}{Phoenix, AZ}}
\iftoggle{superInfo}{\expExtra{%
  \textit{Supervisor:} Frank J.\ Grimmelmann, (623)~551-1526, \mailto{fgrimmelmann@actraaz.org}}}{}


\begin{rExperience}

  \item Designed an operationalized workflow for Arizona Infragard member organizations to share \textbf{threat intelligence} through a common \textbf{STIX/TAXII} platform.

  % \item Created a presentation introducing the basics and benefits of {\bf STIX/TAXII}.

\end{rExperience}


  
% Define the fields for this entry
\def\PositionTitle{Teaching Assistant (Instructor of Record)}
\def\PositionPeriod{Aug 2011 -- Dec 2014} % Remember to use -- between dates
\def\OrgName{Arizona State University}
\def\OrgLocation{Tempe, AZ}


\ifthenelse{\boolean{titleOnly}}{
  \begin{rExperienceHeader}{\PositionTitle}{\PositionPeriod}{\OrgName}{\OrgLocation}
  \end{rExperienceHeader}
}{
\begin{rExperienceBullets}{\PositionTitle}{\PositionPeriod}{\OrgName}{\OrgLocation}

  \item CSE 465 Information Assurance: Fall 2014.

  %\item FSE 100 Introduction to Engineering: Fall 2013, Spring 2013, Fall 2012, Spring 2012, and Fall 2011.
  % \item FSE 100 Introduction to Engineering: Fall 2011, Spring 2012, Fall 2012, Spring 2013, and Fall 2013.
  \item FSE 100 Introduction to Engineering: Fall 2011 -- Fall 2013 (5 semesters).

\end{rExperienceBullets}

}


  
% Define the fields for this entry
\expPosTitle{Student Trainee (Civilian)}
\expPosPeriod{Jun 2013 -- Aug 2013, Jun 2014 -- Aug 2014} % Remember to use -- between dates
\expOrgName{US Army}
\expOrgLoc{\iftoggle{fullAddress}{310R Chamberlain Avenue, Ft.\ Meade, MD 20755}{Fort Meade, MD}}
\expExtra{\threecol{\textit{Grade}: GG-09 Step 1}{\textit{Service}: Excepted}{\textit{Tenure}: Permanent}%
  \iftoggle{superInfo}{%
    \\ \textit{Supervisor:} Gregory Platt (\href{mailto:gregory.a.platt.civ@mail.mil}{gregory.a.platt.civ@mail.mil})}{}%
}


\begin{rExperience}

  \item Summer internships in connection with DoD IASP scholarship.

\end{rExperience}


  
% Define the fields for this entry
\expPosTitle{Graduate Student Summer Intern}
\expPosPeriod{May 2011 -- Jul 2011} % Remember to use -- between dates
\expOrgName{Sandia National Laboratories}
\expOrgLoc{\iftoggle{fullAddress}{1515 Eubank Blvd SE, Albuquerque, NM 87123}{Albuquerque, NM}}
\iftoggle{superInfo}{\expExtra{%
  \textit{Supervisor:} Karen Shanklin, (505)~235-5860, \mailto{klshank@sandia.gov}}}{}


\begin{rExperience}

  \item Helped design a dynamic malware analysis framework built on \textbf{OpenStack}, allowing incident responders to
    define customizable analysis environments and use arbitrary analysis tools for triage or manual analysis.

  \item Wrote \textbf{Python} scripts to automate the setup process for using a SheevaPlug computer as a wireless intrusion detection agent running \textbf{Kismet}.

\end{rExperience}


  % \input{sections/exp_usu_web}


\end{rSection}

%----------------------------------------------------------------------------------------
%	PUBLICATIONS SECTION
%----------------------------------------------------------------------------------------

\begin{rSectionHeading}{Publications (Selected)}
\end{rSectionHeading}
\mybibliography{}{}{major}

%----------------------------------------------------------------------------------------
%	TECHNICAL STRENGTHS SECTION
%----------------------------------------------------------------------------------------

\begin{rSection}{Technical Strengths \& Qualifications}

\begin{tabular}{ @{} >{\bfseries}l @{\hspace{6ex}} l }
Research Interests & Digital Forensics, Internet of Things, Cloud Computing \\
Programming Languages & Python, C/C\verb|++|, HTML, CSS, \LaTeX \\
Forensic Tools & FTK, Sleuth Kit \& Autopsy, dd, HxD, etc.\\
Protocols \& APIs & JSON, XML, REST, STIX/TAXII \\
Network Administration/Security & OpenVPN, Wireshark, ufw, Apache, lighttpd \\
Operating Systems & Windows, Linux, Chrome OS \\
Databases & MySQL, SQLite
\end{tabular}

\end{rSection}

%----------------------------------------------------------------------------------------
%	SCHOOL PROJECTS SECTION
%----------------------------------------------------------------------------------------

%\begin{rSection}{Relevant School Projects}
%\begin{list}{$\cdot$}{\leftmargin=0em} % \cdot used for bullets, no indentation
%   \itemsep -0.5em
%
%	\item Implemented \textbf{Python} scripts that interpreted MBRs and boot sectors for FAT file systems.
%
%	\item Wrote a program in \textbf{Python} to scan \textbf{C} and \textbf{C++} source files for commonly used but insecure function calls, which then suggested to the user more secure yet equivalent library functions.
%
%	%\item Implemented a ``robot'' for the Google Wave platform that translated hex and binary to ASCII as well as interpreted MBRs and VBRs. See http://code.google.com/p/forensie/.
%
%	\item Created a web-based interactive learning module designed to teach basic principles of password strength and symmetrical encryption using \textbf{Python}, \textbf{JavaScript}, and \textbf{Ajax}. \href{http://csilm.usu.edu/~securityninja/}{http://csilm.usu.edu/\textasciitilde securityninja/}.
%
%	%\item Wrote a program in \textbf{C++} that accepted a file containing cipher text created with a symmetrical encryption algorithm and attempted to discover the original key and plain text.
%
%	%\item Demonstrated a buffer overflow attack on a \textbf{C++} program using \textbf{gdb}.
%
%	%\item Configured a \textbf{Linux} machine to be a \textbf{DNS} server, a \textbf{DHCP} server, and a web server that used \textbf{SSL} and authentication via \textbf{htaccess} rules.
%
%	%\item Set up a basic logging system in \textbf{Linux} with \textbf{Snort} and \textbf{syslog}.
%
%	%\item Created a map of local wireless networks by war-walking while running \textbf{NetStumbler}.
%
%	%\item Analyzed Internet Explorer vulnerabilities using Security Innovation's \textbf{Holodeck} program.
%
%\end{list}
%
%\end{rSection}

%----------------------------------------------------------------------------------------
%	AWARDS SECTION
%----------------------------------------------------------------------------------------

\begin{rSection}{Awards and Activities}

  \begin{rBulletList}

  \item DoD Information Assurance Scholarship Program (IASP) Recipient (4 years) \hfill \emph{2012 -- 2016}

  \item Inducted into Eta Kappa Nu (HKN) Engineering Honors Society \hfill \emph{Nov 2010}

  \item Team Leader --- ASU team in the UCSB International CTF \hfill \emph{2009, 2010, 2014, 2015}

  \item Eagle Scout, Boy Scouts of America \hfill \emph{2002}

  \item Tallest Graduate Student at ASU \hfill (not an actual award)

  \end{rBulletList}

\end{rSection}

%----------------------------------------------------------------------------------------
%	EXAMPLE SECTION
%----------------------------------------------------------------------------------------

%\begin{rSection}{Section Name}

%Section content\ldots

%\end{rSection}

%----------------------------------------------------------------------------------------

\end{document}
