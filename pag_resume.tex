%%%%%%%%%%%%%%%%%%%%%%%%%%%%%%%%%%%%%%%%%
% Medium Length Professional CV
% LaTeX Template
% Version 2.0 (8/5/13)
%
% This template has been downloaded from:
% http://www.LaTeXTemplates.com
%
% Original author:
% Trey Hunner (http://www.treyhunner.com/)
%
% Important note:
% This template requires the resume.cls file to be in the same directory as the
% .tex file. The resume.cls file provides the resume style used for structuring the
% document.
%
%%%%%%%%%%%%%%%%%%%%%%%%%%%%%%%%%%%%%%%%%

%----------------------------------------------------------------------------------------
%	PACKAGES AND OTHER DOCUMENT CONFIGURATIONS
%----------------------------------------------------------------------------------------

\documentclass{resume} % Use the custom resume.cls style

\usepackage[left=0.75in,top=0.6in,right=0.75in,bottom=0.6in]{geometry} % Document margins

\usepackage{multicol} %columns for lists
\usepackage{paralist} %lists without indentation
\usepackage{hyperref} %links - http and mailto
\hypersetup{hidelinks} %hide boxes around links
%check out other hyperref options

\setlength\multicolsep{0pt} %remove space before and after multicol

\def\sectionskip{\smallskip} % The space after the heading section
\def\nameskip{\medskip} % The space after your name at the top
\def\addressskip{} % The space between the two address (or phone/email) lines

\name{Justin W. Paglierani} % Your name
\address{6551 E. Preston St \\ Mesa, AZ 85215} % Your address
\address{\href{https://jpaglier.blogspot.com}{https://jpaglier.blogspot.com} \\ \href{https://www.bitbucket.org/jpaglier}{https://www.bitbucket.org/jpaglier}} %blog and bitbucket
\address{(480)~233~-~9666 \\ \href{mailto:jpaglier@asu.edu}{jpaglier@asu.edu}} % Your phone number and email

\begin{document}

%----------------------------------------------------------------------------------------
%	EDUCATION SECTION
%----------------------------------------------------------------------------------------

\begin{rSection}{Education}

%had to add raggedright here for some reason because it didn't like the thesis line being added
\textbf{\raggedright{Arizona State University, Ira A. Fulton School of Engineering\\}}
M.S. in Computer Science (Information Assurance)\hfill GPA: 3.97 \hfill \emph{December 2013}\\
%thesis, flush with the right
{\small\raggedleft{\textbf{Thesis:} ``A Framework for Extended Acquisition and Uniform Representation of Forensic Email Evidence''\\}}
B.S. in Computer Science (Information Assurance)\hfill \emph{January 2011}

\textbf{Relevant Coursework}
\begin{multicols}{2}
	\begin{asparaitem} \itemsep1pt \parskip0pt \parsep0pt
		\item Computer System Security
		\item Data and Information Security
		\item Software Security
		\item Advanced Computer Network Security
		\item Computer and Network Forensics
		\item Embedded Operating System Internals 
		\item Advanced Operating Systems
	\end{asparaitem}
\end{multicols}

\end{rSection}


%----------------------------------------------------------------------------------------
%   COURSE PROJECTS SECTION - THIS PROBABLY WONT LIVE HERE TOO LONG
%----------------------------------------------------------------------------------------

\begin{rSection}{Course Projects}

\begin{asparaitem} \itemsep1pt \parskip0pt \parsep0pt
	\item \href{https://bitbucket.org/jpaglier/cse536-kernel-dev}{Distributed messaging service with accompanying kernel-level transport protocol \hfill (Linux, C)}
	\item Web-based document management system \hfill (Java, Struts2, Tomcat 6, Linux, MySQL)
	\item Cloud-based file sharing protocol and application \hfill (Linux, Python)
	\item \href{https://bitbucket.org/jpaglier/2012-spring-cse494-598_g-project}{Volume and file system analysis tool with an emphasis on FAT systems \hfill (Python)}
	\item Build a secure network containing Apache and bind servers \hfill (iptables, Squid, Snort)
	\item Database system supporting access control and k-anonymity functionality \hfill (Java, MySQL)
	\item Smart sprinkler controller system using an embedded platform \hfill (Linux, cron, Apache, C/C++)
\end{asparaitem}

\end{rSection}


%----------------------------------------------------------------------------------------
%	WORK EXPERIENCE SECTION
%----------------------------------------------------------------------------------------

\begin{rSection}{Experience}

\begin{rSubsection}{\href{http://www.frbsf.org/}{Federal Reserve Bank of San Francisco}}{January 2014 - Present}{Information Security Analyst II, Local Incident Response Team}{San Francisco, CA}

	\item Nothing yet
	
\end{rSubsection}

%------------------------------------------------

\begin{rSubsection}{\href{http://sefcom.asu.edu/}{SEFCOM, Arizona State University}}{January 2011 - December 2013}{Research Assistant}{Tempe, AZ}

	\medskip %multicols breaks the spacing for some reason
	\begin{multicols}{2}
		\item Develop novel e-mail forensics methodologies
		\item Conduct  analysis of Android malware
		\item Create a taxonomy of Android malware
		\item Deploy and manage a cloud for use in a \\ collaborative forensics platform (OpenStack)
		\item Prepare  and deliver weekly presentations
	\end{multicols}

\end{rSubsection}

%------------------------------------------------

\begin{rSubsection}{National Incident Response Team of the Federal Reserve System}{May 2013 - August 2013}{Intern}{San Francisco, CA}

	\item Develop and test a web-based wrapper for penetration testing tools used to provide a middle layer between tools and the functions they perform \hfill (Django)

\end{rSubsection}

%------------------------------------------------

\begin{rSubsection}{\href{http://www.federalreserve.gov/}{Board of Governors of the Federal Reserve System}}{May 2012 - August 2012}{IT Intern, Messaging \& Mobile}{Washington, DC}

	\item Develop/test/document an e-discovery tool. (Ruby, CoffeeScript)	
	\item Document/test a mobile device management platform. (MobileIron, Android, iPhone)

\end{rSubsection}

\begin{rSubsection}{\href{http://www.onsemi.com/}{ON Semiconductor}}{May 2010 - January 2011}{Computer Science Intern, Global Reliability}{Phoenix, AZ}

	\item Assist in testing and development of Reliability Management System (Java EE)

\end{rSubsection}

\end{rSection}

%----------------------------------------------------------------------------------------
%	TECHNICAL STRENGTHS SECTION
%----------------------------------------------------------------------------------------

\begin{rSection}{Technical Strengths}

\begin{tabular}{ @{} >{\bfseries}l @{\hspace{6ex}} l }
Programming Languages & Java, Python, C/C++, Ruby \\
Design Patterns & MVC, Service Locator, etc. \\
Protocols \& APIs & XML, JSON, REST \\
Network Administration/Security & Wireshark, iptables, Snort, etc. \\
Operating Systems & Linux, Windows \\
Databases & MySQL, SQLite \\
Tools & Git, Vim
\end{tabular}

\end{rSection}

%----------------------------------------------------------------------------------------
%	PUBLICATIONS SECTION
%----------------------------------------------------------------------------------------

\begin{rSection}{Publications}
\textbf{Published:}
\href{http://eudl.eu/doi/10.4108/icst.collaboratecom.2013.254125#!}{``Towards Comprehensive and Collaborative Forensics on Email Evidence''}, In Proceedings of \href{http://collaboratecom.org/2013/show/home}{9th IEEE International Conference on Collaborative Computing: Networking, Applications and Worksharing (CollaborateCom)}, 2013.
\end{rSection}

%----------------------------------------------------------------------------------------
%	AWARDS SECTION
%----------------------------------------------------------------------------------------

\begin{rSection}{Awards and Activities}

\begin{asparaitem} \itemsep1pt \parskip0pt \parsep0pt
	\item National Science Foundation Scholarship for Service \hfill January 2012 - December 2013
	\item \href{http://asulug.org/2013/09/installfest-fall-2013-results/}{\($1$st\) place: State Farm’s ASU Linux User Group CTF \hfill September 2013}
	\item \href{https://code.djangoproject.com/ticket/20760}{Accepted patch: Timing attack in Django authentication module (Ticket \#20760)} \hfill July 2013
	\item Participant –-- UCSB International CTF –-- Arizona State University Team \hfill December 2010
	\item Undergraduate Teaching Assistant – CSE 100 Introduction to Java Programming \hfill Fall 2009
\end{asparaitem}

\end{rSection}

%----------------------------------------------------------------------------------------
%	EXAMPLE SECTION
%----------------------------------------------------------------------------------------

%\begin{rSection}{Section Name}

%Section content\ldots

%\end{rSection}

%----------------------------------------------------------------------------------------

\end{document}
