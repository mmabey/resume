%----------------------------------------------------------------------------------------
%	EDUCATION SECTION
%----------------------------------------------------------------------------------------

\begin{rSection}{Education}

\textbf{PhD Computer Science --- Information Assurance}\hfill \emph{(expected) May 2017}\\
Arizona State University \hfill {Tempe, AZ}

\textbf{M.S. Computer Science --- Information Assurance}\hfill \emph{Aug 2011}\\
Arizona State University \hfill {Tempe, AZ}\\
\textit{Committee:} Gail-Joon Ahn (Chair), Stephen S. Yau, Dijiang Huang\\
\textit{Thesis:} \href{http://repository.asu.edu/attachments/56996/content/Mabey_asu_0010N_10959.pdf}{Collaborative Digital Forensics: Architecture, Mechanisms, and Case Study}\\
\begin{quote}
  \vspace{-3ex}
  In order to catch the smartest criminals in the world, digital forensics examiners need a means of collaborating and sharing information that is not prohibitively difficult and that complies with standard operating procedures and the rules of evidence. In this work I present the design and implementation of the Collaborative Forensic Framework (CUFF) which is designed to include collaboration-facilitating components. I also give the details of a realization of CUFF, which uses a combination of Java, the Google Web Toolkit, Django with Apache for a RESTful web service, and an Ubuntu Enterprise Cloud using Eucalyptus.
\end{quote}

\textbf{B.S. Computer Science --- Information Systems}\hfill \emph{May 2009}\\
Utah State University \hfill {Logan, UT}

\end{rSection}
