% !TEX root = ../CV_Mike_Mabey.tex

%----------------------------------------------------------------------------------------
%	EDUCATION SECTION
%----------------------------------------------------------------------------------------

\begin{rSection}{Education}

\textbf{Ph.D.\ Computer Science --- Information Assurance}\hfill \emph{Dec 2017}\\% Dec 11, 2017
Arizona State University\iftoggle{showGPA}{ | GPA 3.76}{} \hfill %
\iftoggle{fullAddress}{{1151 S Forest Ave \#140, Tempe, AZ 85281}\\}{{Tempe, AZ}\\}
\textit{Committee:} Gail-Joon Ahn (Co-Chair), Adam Doup\'{e} (Co-Chair), Stephen S.\ Yau, Jooyung Lee, Ziming Zhao\\
% \textit{Research Topics:} Web environment forensics, Email forensics, Chrome OS forensics\\
\textit{Dissertation:} \href{https://mikemabey.com/Dissertation_Mabey.pdf}{Forensic Methods and Tools for Web
Environments}
\begin{CVonly}\begin{quoting}

  The Web is one of the most exciting and dynamic areas of development in today's technology. However, web environments
  also present a set of new challenges for digital forensic examiners, making their jobs even more difficult. In my
  dissertation, I present (1)~a framework for web environment forensics, which gives examiners a method for how to
  approach web-based evidence; (2)~a method to identify extensions installed on encrypted web thin clients without
  breaking the encryption; and (3)~an approach to reconstructing the timeline of events on encrypted web thin clients by
  using service provider APIs as a proxy for directly analyzing the device. I also introduce several structured formats
  that I customized to accommodate the unique features of web-based evidence while also facilitating tool
  interoperability and information sharing.

\end{quoting}\end{CVonly}

\textbf{M.S.\ Computer Science --- Information Assurance}\hfill \emph{Aug 2011}\\% 8/5/2011
Arizona State University\iftoggle{showGPA}{ | GPA 3.58}{} \hfill %
\iftoggle{fullAddress}{{1151 S Forest Ave \#140, Tempe, AZ 85281}\\}{{Tempe, AZ}\\}
\textit{Committee:} Gail-Joon Ahn (Chair), Stephen S.\ Yau, Dijiang Huang\\
\textit{Thesis:} \href{https://mikemabey.com/MS_Thesis_Mabey.pdf}{Collaborative Digital Forensics: Architecture,
Mechanisms, and Case Study}%
% \begin{CVonly}\begin{quoting}
  % In order to catch the smartest criminals in the world, digital forensic examiners need a means of collaborating and
  % sharing information that is not prohibitively difficult to use and that complies with standard operating procedures
  % and the rules of evidence. In this thesis I present the design and implementation of the Collaborative Forensic
  % Framework (CUFF) which is a cloud-based forensic analysis environment that facilitates collaboration among examiners
  % and their trusted colleagues. CUFF makes a positive impact on forensic examiners' efficiency by helping them leverage
  % their network of subject matter experts while also tapping into the scalability and power of the cloud.
% \end{quoting}\end{CVonly}

\textbf{B.S.\ Computer Science --- Information Systems}\hfill \emph{May 2009}\\% 5/2/2009
Utah State University\iftoggle{showGPA}{ | GPA 3.25}{} \hfill %
\iftoggle{fullAddress}{{1600 Old Main Hill, Logan, UT 84322-1600}}{{Logan, UT}}

\end{rSection}
