%----------------------------------------------------------------------------------------
%	SCHOOL PROJECTS SECTION
%----------------------------------------------------------------------------------------

\begin{rSectionHeading}{Relevant School Projects}
  \begin{rBulletList}

    \item Implemented \textbf{Python} scripts that interpreted MBRs and boot sectors for FAT file systems.

    \item Wrote a program in \textbf{Python} to scan \textbf{C} and \textbf{C++} source files for commonly used but insecure function calls, which then suggested to the user more secure yet equivalent library functions.

    \item Implemented a ``robot'' for the Google Wave platform that translated hex and binary to ASCII as well as interpreted MBRs and VBRs. See http://code.google.com/p/forensie/.

    \item Created a web-based interactive learning module designed to teach basic principles of password strength and symmetrical encryption using \textbf{Python}, \textbf{JavaScript}, and \textbf{Ajax}. \href{http://csilm.usu.edu/~securityninja/}{http://csilm.usu.edu/\textasciitilde securityninja/}.

    \item Wrote a program in \textbf{C++} that accepted a file containing cipher text created with a symmetrical encryption algorithm and attempted to discover the original key and plain text.

    \item Demonstrated a buffer overflow attack on a \textbf{C++} program using \textbf{gdb}.

    \item Configured a \textbf{Linux} machine to be a \textbf{DNS} server, a \textbf{DHCP} server, and a web server that used \textbf{SSL} and authentication via \textbf{htaccess} rules.

    \item Set up a basic logging system in \textbf{Linux} with \textbf{Snort} and \textbf{syslog}.

    \item Created a map of local wireless networks by war-walking while running \textbf{NetStumbler}.

    \item Analyzed Internet Explorer vulnerabilities using Security Innovation's \textbf{Holodeck} program.

  \end{rBulletList}

\end{rSectionHeading}
